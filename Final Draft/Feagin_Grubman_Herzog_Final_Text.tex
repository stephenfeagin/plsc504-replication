\documentclass[12pt]{article}
\usepackage{fullpage}
\usepackage[american]{babel}
\usepackage[autostyle, english=american]{csquotes}
\usepackage{xpatch} 
\usepackage{tabularx}
\usepackage{graphicx}
\usepackage{multirow}
\usepackage{rotating}
\usepackage[colorlinks=true, citecolor=blue]{hyperref}
\usepackage{amsfonts,amsmath,bm} 


% !BIB program = biber
\usepackage[authordate, backend=biber,
	sorting=nyt,
	sortcites=true,
	isbn=false,
	url=false,
	doi=false,
	language=american
	]{biblatex-chicago}

\addbibresource{WeeksRepl.bib} % This line links to the bibliography I uploaded to the shared box folder. Feel free to make any additions you want, but make sure to re-upload it to box or make sure it syncs. When you update a bibliography file, you usually have to delete all the auxiliary files (everything except .tex and .pdf), which can be done manually or with the "Trash Aux Files" button on the LaTeX log window. Then you run the document with LaTeX once, then change it to Bibtex (the dropdown menu is right beside the "Typeset" button up top), run it once with Bibtex, then run it again with LaTeX twice.


\title{War, Authoritarian Institutions, and the Democratic Peace \\ \bigskip \Large PLSC 504 Replication Project}
\author{Stephen Feagin \and Nate Grubman \and Stephen Herzog}


\begin{document}
\maketitle

\section{Authoritarian Institutions as Causes}

\par In a series of articles and a recent book, Jessica L. \textcites{weeks:2008}{weeks:2012}{weeks:2014} presents a theory of war that transcends the boundary between international relations and comparative politics. Criticizing the international relations literature on domestic institutions as causes of war---often operationalized by drawing a crude division between democratic and nondemocratic regimes---Weeks argues that the authoritarian world is not as uniformly careless and belligerent as the democratic peace suggests. 

\par Building upon the literature on comparative authoritarianism, Weeks argues that certain authoritarian regimes have institutions that constrain leaders while others do not. In her \cite*{weeks:2008} piece, she uses Barbara Geddes's \parencite*{geddes:2003} typology of regimes, arguing that personalist regimes are less constrained than others. Later, she refines her typology, arguing that there are two dimensions that matter: whether or not the regime is personalist determines the ability of the ruling coalition to constrain, and whether or not the military plays a significant role in politics determines the preferences of the ruling coalition with respect to using that constraint to avoid war \autocite[329-330]{weeks:2012}. Using the Correlates of War (COW) data that have been critical to quantitatively establishing the democratic peace, Weeks observes that certain authoritarian institutional arrangements indeed correlate with a higher propensity to initiate violent conflict. 

\par That authoritarian institutions matter and that not all authoritarian regimes are alike is of course not a new argument. Indeed, examining the effects of various institutional arrangements motivates much of the research on comparative authoritarianism. Some have argued that authoritarian institutions cause greater stability \parencite{gandhi:2007}, economic growth \parencite{gandhi:2008}, better corporate governance \parencite{jensen:2014}, and greater public-goods provision \parencite{miller:2015}. Furthermore, the constraining effect of legislative institutions on otherwise bellicose leaders is a common mechanism used to explain the democratic peace. 

\par Yet, the correlation between electoral autocracy and peace raises as many questions as it answers. First, taking authoritarian institutions as causes of political outcomes is problematic because institutions are themselves the products of political conflict and may reflect underlying power structures. As Carles Boix and Milan Svolik argue, the literature on authoritarian institutions is troublesome because \enquote{Few authors clearly identify how institutions [facilitate authoritarian governance], why the same results could not be accomplished without them, and why they are adopted in some cases but not others} \parencite*[301]{boix:2013}. Or, as Thomas Pepinsky puts it, \enquote{The central problem for institutional analysis is how to reconcile the fact that institutions are political creations with the observation that they correlate very well with outcomes that matter to political scientists\ldots [B]ecause institutions are political creations, analysts must distinguish between institutions as \textit{causes} and institutions as \textit{epiphenomena}} \parencite*[648]{pepinsky:2014}. To understand the causal effects of institutions, we must understand the processes that govern their selection. 

\par Second, if more institutionalized authoritarian regimes are indeed more peaceful, it is not necessarily clear why this is the case. The challenge of explaining Weeks's finding is similar to the challenge regarding the democratic peace that caused Charles \textcite{lipson:2005} to joke, ``We know it works in practice. Now we have to see if it works in theory!'' While Weeks offers a mechanism by which electoral authoritarianism may lead to more cautious foreign policy, she does not test this mechanism or account for other possibilities. While much of the literature on authoritarian institutions has focused on their ability to shape cooperation and conflict within the authoritarian ruling coalition, other studies have found that these institutions help autocrats by creating institutionalized mechanisms by which limited opposition can be expressed and addressed. It remains unclear if electoral authoritarian regimes are more peaceful because they have ruling parties that constrain dictators (as Weeks suggests) or because they feature more robust mechanisms by which the public may have an effect on policy. With both the mechanisms underlying the democratic peace and the effects of authoritarian institutions still contested scholarly terrain, there is much at stake.

\par In this paper, we examine the causal explanation that Weeks offers to account for the correlation between authoritarian institutions and peace. We test whether the electoral strength of the opposition and the relative freedom of the media may serve as constraints on belligerence. We find limited support for Weeks's mechanism in light of these alternative constraints. Specifically, the presence of a large electoral opposition seems to correlate with a higher propensity to go to war. This suggests that Weeks is correct to locate her constraint in the ruling party. Meanwhile, the effect of media freedom is conditional upon regime type. In regimes that do not have ruling parties, media freedom may serve as a substitute constraint. 

\section{Weeks's Theory}

\par Weeks tests her theory on a sample of dyads dating from 1946 until 1999. She codes each country according to its regime type, including democracies and four types of authoritarianism varying by level of institutionalization and military involvement in politics. Weeks's primary explanatory variable is \textit{constraints}---the presence of an audience to whom the authoritarian regime is accountable. Yet she uses a quite coarse measure of these constraints, creating an index of variables from Geddes's regime type data \autocite[Appendix A]{geddes:2003}. Constraint collapses eight regime characteristics upon each other, ranging from the dictator's personal control of patronage and security forces to the presence of ethnic violence. In the figures below, we chart the prevalence of each regime type over time. Machines are highly institutionalized governments with civilian leadership, while juntas are highly institutionalized regimes with military leadership. Bosses are personalist regimes led by civilians, whereas strongmen are personalist military regimes. Below are figures tracking the prevalence of each regime type over time, as well as the number of country-years covered by each regime type. Interestingly, machines cover more country-years than do any other regime type, but the number of machines at any period in time is few. This is unsurprising, given the finding that institutionalized authoritarian regimes are more durable, but it also suggests that the set of countries which Weeks finds to be more peaceful is relatively small.\footnote{The appendix includes the full list of countries that fall under each regime type. Weeks's finding is robust to the dropping of any one country.} 

\begin{figure}
\includegraphics[width=\textwidth]{rtype_plots.pdf}
	\caption{Counts of regime type in the Weeks (2012) dataset}
\end{figure}

\par Weeks runs a series of logistic regressions on her dataset. The publication of her dataset, as well as the accompanying ``do" files, facilitates easy replication of her results in \textit{Stata.} Of her four models, we translate the two attributed to her main findings into R and can successfully reproduce her results. We omit her fixed-effects models, siding with \textcite[249]{dafoe:2011} who notes that ``The consensus seems to be that for fixed effects, the econometric cure is worse than the disease.''\footnote{These are replicable in Stata but seem to require tremendous computing power in R.} Figure 2 is a coefficient plot corresponding to the first model in the table of her paper. As this figure shows, Weeks finds that after controlling for other causes of war, ``machine" regimes are no more likely than democracies to initiate international conflict.

\subsection{Mechanisms of Constraint}

\par Weeks theorizes a mechanism by which electoral authoritarian regimes may make dictators less likely to go to war. In an earlier paper, she argues that (most) authoritarian leaders do not live in a world free of domestic constraints: ``[E]ven without democratic institutions, autocratic leaders depend on the support of domestic groups to survive in office" \parencite*[38]{weeks:2008}. So, dictators, like democratic leaders, have audiences, albeit not very inclusive ones. Weeks argues that the willingness and ability of these audiences to credibly threaten dictators with serious consequences for policy missteps varies according to the institutional arrangement. But who belongs to the authoritarian audience, and how does institutional variation explain variation in audience power?

\par According to Weeks, authoritarian institutions constrain dictators not because of their empowerment of the public or their creation of a channel for peaceful opposition challenges, but rather because they empower members of the dictator's own ruling coalition. As Weeks argues, \enquote{nondemocracies vary greatly in the extent to which \textit{regime insiders} can coordinate to punish the leader, rendering regime elites an effective audience} \parencite*[40, emphasis in original]{weeks:2008}. As she writes more recently, \enquote{government insiders serve as powerful domestic audiences in nonpersonalist regimes} \parencite[330]{weeks:2012}. She argues that electoral authoritarianism, or at least the absence of personalism, endows members of the ruling coalition with greater independence from the leader and greater ability to check the leader's power. Conversely, control of the security apparatus and access to high office allows the personalist dictator to monitor and punish elites' attempts to coordinate against him. In regimes with strong institutions, insiders in high office can be more assured of their prospects for safety and prosperity in case the executive is removed, which provides them a degree of insulation and independence from the personal whims of the dictator.

\par Weeks's measurement of personalism, an index of variables from Geddes's \cite*{geddes:2003} dataset, reflects her emphasis on control over the security apparatus, high office, and coordination among regime insiders. While her index does produce substantial and significant predictions of belligerence, it is difficult to pinpoint whether Weeks's theory---independent regime insiders constrain dictators---is true, or whether other, collinear, mechanisms explain why less personalistic regimes tend to be less pugnacious. Indeed, while Weeks's theory denies a place for actors outside the ruling circle, many of the same regimes that feature more independent institutions are also characterized by mechanisms for peaceful opposition, and often a more free media. It is plausible that regime outsiders may be driving the correlation that Weeks finds.

\begin{figure}
	\includegraphics{reg_1_2}
	\caption{Coefficient plot of Weeks's original results: MID initiation regressed on categorical regime type, with the full set of covariates. She presents a clear negative relationship between machine regimes and conflict initiation.}
\end{figure}

\par There are certainly reasons to believe that the strength of the formal and public opposition may be a significant constraint, and one that works independent of the strength of regime insiders. Authoritarian regimes with multiparty legislatures are more likely to sign international agreements \autocite{vreeland:2008} and attract greater domestic investment \autocite{gehlbach:2012} than are single-party regimes. Strong electoral performance by the opposition leads to policy concessions by the regime and a reduction in military spending \autocite{miller:2015}. Jennifer \textcite{gandhi:2008} likewise finds that opposition performance is an important predictor of policy outcomes.

\par These results may explain an important part of why electoral authoritarian regimes are more durable than non-electoral regimes. They can adjust policy to the strength of the opposition, preventing a buildup of tension and unrest among a dissatisfied populace \autocites(see )(){he:2011}{he:2013}{truex:2014}. It could be, then, that closer elections and greater opposition participation in the legislature should cause dictators to be more cautious in deciding to use war, for the same mechanism of threat of punishment that Weeks argues. Thus, we hope to disentangle the effect of authoritarian insiders and opposition in constraining the executive. Meanwhile, it is also possible that the public can act as a constraint on the executive in much the same way that legislative opposition can. If this were the case, we would expect that in conditions of greater media freedom, the dictator would be less likely to initiate violent conflict. It is quite possible that the democratic dynamics described by Potter and Baum \autocite*{potter:2014} also obtain in autocracies. To disentangle these three potential mechanisms, we intend to measure the correlation between conflict initiation and measures of two different constraints. The first offers an opportunity to adjudicate between Weeks's theory and idea that a strong electoral opposition can constrain a dictator:

\begin{equation}
aggression = \beta_0 + \beta_1*ruling party seats + \beta_i*covariates
\end{equation}

\par If the coefficient on ruling party seats is positive, it suggests that Weeks's mechanism may be correct. If it is negative, it suggests that opposition-party strength may be the true constraint binding the leader. The second equation offers an opportunity to evaluate whether the media may serve as a constraint. 

\begin{equation}
aggression = \beta_0 + \beta_1*media freedom + \beta_i*covariates
\end{equation}

\par If the coefficient on media freedom is positive, then that offers evidence that the media may serve as an alternative constraint to that theorized by Weeks. 

\section{Measures}

\subsection{Formal Opposition}

\par To adjudicate between Weeks's mechanism and that of a strong electoral opposition, we use the Authoritarian Regimes Dataset (ARD); \autocites{hadenius:2007}{wahman:2013}. ARD covers all countries of the world during the period 1972--2010, overlapping with Weeks's dataset for 27 years. The dataset includes two measures of the size of the ruling party. \textit{Partsz} measures the percentage of seats held by the majority party in the legislature.\footnote{For bicameral legislatures, only the lower house is counted.}  \textit{Partsz1} is a similar measure, but it codes as zero legislatures in which one party holds all of the seats. A score of 0.5 for $partsz$ or $partsz1$ signifies that the largest party in the legislature holds half of its seats. To the extent that the strength of a ruling party covaries with its electoral performance, high scores of \textit{partsz} signify a ruling party that is electorally strong. Low scores signal one that faces robust competition.  

\begin{figure}
	\includegraphics{hist_5_1}
	\caption{Histogram of the relative size of the largest party across authoritarian regimes. The \textit{partsz} variable calculates the proportion of legislative seats held by the largest party.}
\end{figure}

\begin{figure}
	\includegraphics{partsz_vioplot}
	\caption{This violin plot shows the substantial variance in largest party size across authoritarian regime types. The width of each violin is proportional to the density of observations at each level.}
\end{figure}

\subsection{Public Opposition}

\par To measure media freedom, we use the Global Media Freedom Dataset, 1948-2012 \parencites{whitten-woodring:2015}. This widely used cross-national dataset uses expert coding to characterize countries based on their relative level of media freedom on a four-point scale. The dataset conceptualizes media freedom as ``an environment in which journalists are able to safely criticize political and economic elites at both the national and local levels''\parencite{whitten-woodring:2015}. This dataset measures media freedom along a continuum of 0--4. A score of ``0" indicates no nationwide media (effectively missing data),\footnote{There are only two of these cases in the entire dataset, which we drop from our analysis.} while a score of ``1" represents free and transparent media. Meanwhile, a score of ``2" entails imperfectly free media with some level of independent journalism existing alongside restrictions on media outreach. Scores of ``3" and ``4" are indicative of a lack of free media, whereby criticism of official policy is not permitted by the state, and the government controls media output through official or unofficial channels. This dataset is the only one of its type that includes all countries and uses duplicative coding by trained academic coders, rather than by researchers associated with think tanks or advocacy organizations. Furthermore, it covers a considerably longer timespan than other available media freedom datasets. Coding is based on the ability of the national media to serve as a check on power, which makes it a particularly suitable measure of the type of constraint we are looking to test. High levels of \textit{media score} signify a national media that is not free to criticize the holders of political power. Although levels of democracy correlate with \textit{media score}, the correlation is far from perfect. Below is a figure charting variation in \textit{media score} across Weeks's regime types.   

\begin{figure}
	\includegraphics{hist_media}
	\caption{Histograms of media scores by authoritarian regime type. Note: Higher scores mean greater restriction of media.}
\end{figure}

\section{Analysis}

\par We performed several levels of analysis. First, we replicated Weeks's findings for the set of years covered by our dataset on party size. Then, we performed a series of logistic regressions using Weeks's outcome variable and covariates, but incorporating our explanatory variables. Weeks's outcome variable is $initmid$, a binary variable for whether side A in a dyad initiates a militarized interstate dispute (MID). In her conditioning strategy, Weeks controls for differences in capabilities and relative power that may explain differences in propensities to go to war, adjusting for alternative paths to war, rather than balancing the determinants of the explanatory variable of interest \parencite[127]{morgan:2014}. We employed three strategies to test our new explanatory variables. First, we included our explanatory variables $partsz$ and $media score$ in Weeks's initial models to see whether they would affect the statistical significance of her explanatory variable. Then, we used our explanatory variables as replacements for Weeks's regime variables in the universe of authoritarian regimes. Finally, we performed subgroup analysis, testing whether the effect of our variables changed within each of Weeks's regime types. 

\subsection{Replicating Week's Analysis}

\par Weeks conducts her analysis using two binary logistic regression models aimed at evaluating a country's likelihood--or lack thereof--of initiating a MID (dependent variable: $initmid$) based on its regime type. Her first model (herein, Model 1) is a parsimonious model, which is inclusive of control variables pertaining only to a state's power projection capabilities. Meanwhile, her second full model (Model 2) controls for a larger range of relevant covariates. Standard errors are clustered by directed-dyad. In our replication, we adopt a function for dyadic cluster-robust standard errors, following \textcite{aronow:2015}.

\par We are able to successfully replicate Week's results for two different periods of time. First, her available replication data files enable reproduction of both of her directed-dyad logit models for the time period she analyzes: 1946-1999. And second, we can replicate her findings specifically for the period of 1972-1999 for which we have data to examine formal legislative constraints on executive authority. Weeks's results remain stable across both periods, thereby indicating that by focusing on a narrower period than does Weeks, we are not biased against confirming her mechanism.

\subsection{Formal Constraints Analysis} 

\par After replicating Week's findings, we analyze the effect of $partsz$ on the probability of initiating a MID. Our examination appears to support Weeks's theory that a strong ruling party constitutes a strong constraint on the legislature. Strong opposition electoral performance, by contrast, seems to make the initiation of a MID more likely. In Weeks's parsimonious model, the addition of the \textit{partsz} makes the coefficient on machine positive but not statistically significant, which suggests that even controlling for the size of the largest party in the legislature, machines are no more likely than democracies to initiate MIDs. Further, in her parsimonious model, largest party size has a negative and significant effect on $initmid$ when controlling for regime type. This suggests that, if a dictator's own political party is large and powerful, it can constrain bellicose executive decision-making. However, since this analysis is performed on the full dataset, including both democracies and autocracies, this result is difficult to interpret. Replication of Weeks's full model with the inclusion of \textit{partsz} generates similar findings: the effect of machines is negative but insignificant, whereas party size is both negative and statistically significant.

\par Hadenius and Teorrell code numerous countries as having ``0" or ``1" for party size, which we investigate further. The notion that a country would have a legislature absent of political parties (coding of ``0") is a somewhat counterintuitive phenomenon and may point to an autocrat's desire to ban parties as a source of factionalism and conflict. These datapoints are difficult to interpret and may also, in practice, just be cases of missing data. On the other hand, a coding of ``1" suggests a ``packed house" and existence of a potentially powerful class of elites standing behind the dictator. We eliminate datapoints with codings of ``0" and replicate Weeks's findings again, only to reveal that her main results remain unchanged in both the parsimonious and full models. The size of the largest party continues to have a negative and significant effect on $initmid$, and the correlation between machines and conflict remains insignificant. The findings from these analyses are presented in Tables 1 and 2 below.


% Table created by stargazer v.5.2 by Marek Hlavac, Harvard University. E-mail: hlavac at fas.harvard.edu
% Date and time: Fri, Dec 18, 2015 - 14:48:18
\begin{table}[!htbp] \centering 
  \caption{Categorical Regime Type Variables with Parsimonious Covariates} 
  \label{} 
\begin{tabular}{@{\extracolsep{5pt}}lcccc} 
\\[-1.8ex]\hline 
\hline \\[-1.8ex] 
 & \multicolumn{4}{c}{\textit{Dependent variable:}} \\ 
\cline{2-5} 
\\[-1.8ex] & \multicolumn{4}{c}{Probability of MID Initiation} \\ 
 & Original & With Party Size & With Party Size ${\neq}$ 0 & With Party Size ${\neq}$ {0,1} \\ 
\\[-1.8ex] & (1) & (2) & (3) & (4)\\ 
\hline \\[-1.8ex] 
 Party Size &  & $-$0.342$^{\dagger}$ & $-$0.678$^{**}$ &  \\ 
  &  & (0.146) & (0.231) &  \\ 
  & & & & \\ 
 Party Size (alt) &  &  &  & $-$0.293 \\ 
  &  &  &  & (0.269) \\ 
  & & & & \\ 
 Machine & 0.067 & 0.117 & 0.291 & 0.559 \\ 
  & (0.150) & (0.218) & (0.239) & (0.295) \\ 
  & & & & \\ 
 Junta & 0.600$^{***}$ & 0.551$^{**}$ & 0.317 & 0.388 \\ 
  & (0.179) & (0.189) & (0.313) & (0.320) \\ 
  & & & & \\ 
 Boss & 0.798$^{***}$ & 1.175$^{***}$ & 1.203$^{***}$ & 1.537$^{***}$ \\ 
  & (0.136) & (0.170) & (0.202) & (0.202) \\ 
  & & & & \\ 
 Strongman & 1.086 & 0.793$^{***}$ & 0.896$^{***}$ & 0.973$^{***}$ \\ 
  & (0.159) & (0.203) & (0.269) & (0.270) \\ 
  & & & & \\ 
\hline \\[-1.8ex] 
Observations & 797,312 & 644,253 & 507,795 & 403,097 \\ 
\hline 
\hline \\[-1.8ex] 
\textit{Note:}   & \multicolumn{4}{r}{$^{\dagger} p<0.05$; $^{*} p<0.01$; $^{**} p<0.005$; $^{***} p<0.0001$} \\ 
\end{tabular} 
\end{table} 



% Table created by stargazer v.5.2 by Marek Hlavac, Harvard University. E-mail: hlavac at fas.harvard.edu
% Date and time: Fri, Dec 18, 2015 - 14:49:59
\begin{table}[!htbp] \centering 
  \caption{Categorical Regime Type Variables with Full Covariates} 
  \label{} 
\begin{tabular}{@{\extracolsep{5pt}}lcccc} 
\\[-1.8ex]\hline 
\hline \\[-1.8ex] 
 & \multicolumn{4}{c}{\textit{Dependent variable:}} \\ 
\cline{2-5} 
\\[-1.8ex] & \multicolumn{4}{c}{Probability of MID Initiation} \\ 
 & Original & With Party Size & With Party Size ${\neq}$ 0 & With Party Size ${\neq}$ {0,1} \\ 
\\[-1.8ex] & (1) & (2) & (3) & (4)\\ 
\hline \\[-1.8ex] 
 Party Size &  & $-$0.491$^{***}$ & $-$0.543$^{\dagger}$ &  \\ 
  &  & (0.132) & (0.236) &  \\ 
  & & & & \\ 
 Party Size (alt) &  &  &  & $-$0.098 \\ 
  &  &  &  & (0.287) \\ 
  & & & & \\ 
 Machine & $-$0.496$^{**}$ & $-$0.068 & 0.015 & 0.512 \\ 
  & (0.170) & (0.227) & (0.253) & (0.291) \\ 
  & & & & \\ 
 Junta & 0.464$^{**}$ & 0.424$^{\dagger}$ & 0.139 & $-$0.007 \\ 
  & (0.166) & (0.175) & (0.293) & (0.317) \\ 
  & & & & \\ 
 Boss & 0.613$^{***}$ & 1.151$^{***}$ & 1.111$^{***}$ & 1.213$^{***}$ \\ 
  & (0.146) & (0.164) & (0.202) & (0.207) \\ 
  & & & & \\ 
 Strongman & 0.791 & 0.661$^{***}$ & 0.713$^{**}$ & 0.449 \\ 
  & (0.126) & (0.167) & (0.237) & (0.249) \\ 
  & & & & \\ 
\hline \\[-1.8ex] 
Observations & 797,312 & 542,643 & 425,672 & 335,500 \\ 
\hline 
\hline \\[-1.8ex] 
\textit{Note:}  & \multicolumn{4}{r}{$^{\dagger} p<0.05$; $^{*} p<0.01$; $^{**} p<0.005$; $^{***} p<0.0001$} \\ 
\end{tabular} 
\end{table} 



\par While Weeks's dataset includes both autocracies and democracies alike, her most important contributions pertain to authoritarian regimes. For this reason, we focus here on the effect of \textit{partsz} in authoritarian contexts across machine, junta, strongman, and boss regime types. We control for regime type and drop codings of ``0" from Weeks's analysis. The results show negative, but insignificant effects for party size. The full model, however, suggests negative but more significant effects for the same relationship, providing model-specific support for Weeks's theory. For the full dataset of authoritarian states, there is support for Weeks's theorized mechanism that strong ruling parties constitute strong constraints. However, care must be taken when interpreting these results. While coefficients may be large and significant, the resulting change in estimated likelihood of war is quite small. A Zelig simulation estimated that, among the population of autocracies, a shift in largest party size from one standard deviation below the mean to one standard deviation above the mean reduced the probability of MID initiation by 0.000126. This result is significant ($p<.03$), but miniscule. Similarly, a hypothetical change from a boss to a machine would decrease the dependent variable by 0.000994. This is a particular issue endemic to this type of data: the dependent variable is initiation of a conflict by one partner in a directed-dyad. This type of measurement diminishes the statistical impact of these already rare events.

\begin{figure}
	\includegraphics{coef_partsz_subgroup}
	\caption{This plot shows regression coefficients on the \textit{partsz} variable after excluding cases with no parties.} 
\end{figure}

\par In addition to performing analysis on the full set of authoritarian regimes, we analyzed the effect of party size within each regime type. This analysis is particularly important because the distribution of largest party size differs systematically by regime type. As our violin plot illustrates, machines usually feature a large ruling party, whereas juntas do not. Because machines are less likely to initiate MIDs than are juntas, an analysis that does not distinguish between regime type is likely to find the effect of party to be negative. Furthermore, Weeks argues that the effect of legislatures on executive authority is most pronounced in institutional autocracies rather than in personalist regimes. By this token, if her theory is accurate, we should expect to see the strongest impacts of \textit{partsz} for machines and juntas, with negligible effects for strongmen and bosses. After all, those regimes that have legislatures that are coded as ``rubber stamps'' are coded by Weeks as strongmen or bosses. For juntas, the coefficients are negative and highly significant in all models that we test. This suggests that in authoritarian regimes with relatively meaningful legislatures, strong ruling parties make aggression less likely. 

\par Given that by definition bosses and strongmen are not supposed to have meaningful legislatures, it is a bit surprising that our subgroup analysis yields some statistically significant relationships within these regime types. Without dropping zeros or ones, the result is negative and significant according to both of Weeks's models. Dropping zeros (but not ones), it is negative and insignificant in Model 1 and negative and significant in Model 2. Dropping both zeros and ones, the effect is insignificant in either model. 

\begin{figure}
	\includegraphics{reg_21_22}
	\caption{Plot of coefficients for media score. The first regression estimated the association between media score and MID initiation in the subset comprising both machines and juntas, and the second in the subset of both strongmen and bosses. This allows us to pool the effects of ``constraints," as Weeks identifies them.}
\end{figure}

\subsection{Informal Constraints Analysis} 

\par Our replication of Weeks's analysis with the addition of data on autocratic media freedom supports the notion that independent journalism can constrain executive authority, even in authoritarian regimes. When assessing democracies and autocracies alike, we see that the association between low media freedom and $initmid$ is both positive and significant across Models 1 and 2. This suggests that regimes, which do not permit free and transparent media, are more war-prone and likely to attack other states. Another important finding that emerged is that for both models the effect of machines---the regime type most similar to democracy---on $initmid$ remains negative and statistically significant. Even with the influence of media as a mechanism of constraint, Weeks's core findings remain robust. Analysis across the full set of autocratic regimes reveals mixed effects of media freedom on MID initiation, depending on model specification. For Model 1, we find that a high media score, indicating a lack of free media, correlates with an increased propensity to initiate conflict. The p-value is significant at approximately 0.004. But for Model 2 with its additional covariates, this positive effect is only significant at slightly less than the 80-percent level.

\par Subgroup analysis yields another interesting finding that we had not expected but could be theoretically interesting. The effect of a free media on propensity to initiate conflict is strongest in strongmen and bosses, the regimes believed to lack legislative constraints. Our subgroup analysis displays somewhat surprising results with respect to machines and juntas, the regime types highlighted by Weeks for their institutionalized nature. We find that media has a positive and insignificant effect on $initmid$ in both Weeks's full and parsimonious models for machines and juntas. If Potter and Baum's \autocite*{potter:2014} media freedom findings for democracies have application to authoritarian regimes, we should expect particularly pronounced effects of media freedom on MID initiation in cases where there is formal opposition (i.e., not a one-party state). However, despite varying levels of legislative opposition in states governed by machines and juntas, we essentially observe no effects of media freedom on $initmid$ for these subgroups. This result remains robust when we combine these institutionalized regimes into one group and repeat our analysis.

\par When we combine the personalist regime types into one group for analysis, we observe an effect of media freedom on MID initiation that is highly significant. Although the effect size is small, these findings suggest that in the absence of formal legislative constraints on the autocratic executive, relative freedom of media may substitute as a constraint on the war-making power of the regime. This effect is very statistically significant in bosses but insignificant in strongmen. Although it is possible that the correlation is a statistical anomaly, the idea that media may serve as a substitute constraint is worthy of further investigation.  

\subsection{Determinants of Belligerence}

In order to make a causal argument, we would have to better understand the means by which regimes are selected, electoral opposition is allowed, and media freedom is determined. Weeks does not dedicate much attention to the origins of authoritarian institutions. In her book, she mentions that scholars of the origins of personalistic regimes, such as Milan Svolik, tend to emphasize domestic reasons for the emergence of these institutions \parencite[33--34]{weeks:2014}. Likewise, she cites \cite{svolik:2013}, who finds that interstate warfare does not predict military intervention in politics.  

\subsection{Placebo Test}

\par Although we are able to observe several---even sometimes surprising---correlations in our analysis of formal and informal constraints, the overall data generating process remains a virtual ``black box." What is the data generating process? Why do some countries have junta regimes and others have bosses, for example? And why is it that some autocracies have relatively free media compared to others in their same regime type group? If we are truly to be able to identify causal relationships, the regime type treatment must be conditionally independent of potential outcomes, which are the various propensities to initiate MIDs under different levels of treatment. Unfortunately, this relationship is not something we can test because the historical data does not allow us to observe implications stemming from counterfactual, situational regime types. 

\par For this reason, we conducted a placebo test to determine if other variables that Weeks does not examine illustrate systematic difference between the four types of authoritarian regimes. While Weeks controls for the aggregate capabilities of states, she does not control for levels of natural resources. According to \cite{smith:2005}, meaningful authoritarian institutions are often a substitute for natural-resource wealth, which could be used for patronage. Similarly, \textcite*{egorov:2009} find that oil-poor authoritarian regimes are more likely to adopt free media. Using a new dataset on natural gas by \parencite*{ross:2015} to Weeks's dataset, we test whether conditional on Weeks's covariates, the propensity of being a machine depends on the level of oil and natural gas wealth per capita. We conducting a similar test for the propensity to adopt free media.  

\par The results of our placebo tests cast doubt upon our ability to identify causal relationships in this replication exercise. We find that the effect of oil and natural gas wealth on a country's propensity to be governed by a machine is both negative and highly significant across Weeks's parsimonious and full models. This suggests that a highly institutionalized government may be a substitute for the benefits of patronage. Regimes in countries with considerable oil and gas wealth are usually able to provide subsidies and jobs to their populations, perhaps making them less likely to adopt highly institutionalized forms of government to enforce laws or extract taxation. The relationship is similar with regard to media freedom. Countries that are rich in natural resources are unlikely to have free medias. 

\par Another point of consideration is whether or not oil and natural gas wealth--rather than regime type--makes countries more likely to initiate MIDs. To test this, we control for regime covariates across Weeks's dataset for the full group of autocracies. We observe a negative effect of oil wealth on $initmid$, significant at the 89-percent level, in the parsimonious model and at the 80-percent level in the full model. We can conclude that regimes with oil and natural gas wealth are indeed systematically different than those that lack these energy resources. However, the differences between these regimes types might not be statistically significant with respect to the propensity to go to war ($initmid$). Interestingly, adding oil and natural gas wealth to Weeks's models, however, does not diminish the significance of her findings. With respect to media freedom, the effect of media freedom remains positive in both the parsimonious and full models, but the finding becomes insignificant in the full model when we control for oil and natural gas wealth. 

\par While these placebo tests do not break the statistical findings of Weeks or our findings regarding media freedom, the imbalance that exists with respect to different regimes and different levels of media freedom warrants great caution in making causal claims. Authoritarian countries with meaningful legislatures are very different than those without them in a number of ways that extend beyond the existence of a ruling party. The fact that these institutions seem to correlate with economic growth, stability, investment, and other positive outcomes suggests that causal claims based on observation of authoritarian regime types will be subject to all kinds of confounding.  

\section{Conclusion}

\par In this paper, we find some evidence that authoritarian regimes boasting electorally strong ruling parties are less likely to initiate violent conflict. In these findings, there is some support for Weeks's theorized mechanism, as well as the general argument of many scholars of comparative authoritarianism that the real threat to dictators lies not with the masses but with the elites who staff ruling coalitions. That strong ruling parties seem to constrain dictators more effectively than do strong opposition parties provides more evidence that dictators fear the coup more than the revolution.

\par Second, we have found some tentative evidence that a free media may serve as a substitute constraint in authoritarian countries that lack independent ruling coalitions. Although this finding was a bit surprising and could be anomalous, it points to the need for more research on those authoritarian regimes that do not hold regular, somewhat meaningful elections. Of course, this finding raises questions regarding why some of these countries develop free media while others do not. 

\par Lastly, our placebo test with oil and natural gas data indicates the overall difficulty of obtaining causal inference using large-N, cross-national, time-series data. Of course, omitted variables and confounding factors are always a risk when conducting regression analyses. In this case, we observe that oil and natural gas wealth may indeed play a role in determining regime type and media freedom and thus might be one of the unexpected drivers of our added-variable approach to replicating Weeks's argument. Our results do show that countries of various regime categories may be systematically different than each other along a spectrum of potentially confounding covariates. And without a close understanding of the intricacies of the data generating process, identifying causal relationships can be difficult, if not impossible. However, regardless of confounding, we find that regime types may serve as reliable predictors of the propensity for various formal and informal constraint mechanisms on MID initiation. Making strides in causal inference regarding the relationship between dictators, their institutions, and propensities to go to war will likely require a stronger research design. But the correlations uncovered in this paper at least point the way toward the relationships that require disentangling. 

\pagebreak

\printbibliography
\end{document}